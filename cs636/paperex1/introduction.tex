\section{Introduction}
\label{sec:Introduction}

The future of dependable distributed computing will rely on better
integration of fault tolerance mechanisms at all levels of computer
systems. The programming language run-time system is a natural
location for dependability support. In the long-term, new programming
language semantics with replication and failure handling will help
programmers write fault-tolerant software. However, much more research,
development, and experimentation will be required before new languages
are adopted.  Our belief is that using a high productivity language such as Python~\cite{Python} can allow researchers to explore a larger design space when compared to conventional statically-typed languages.

We have developed the \emph{River} framework to facilitate the
development of reliable distributed Python programs and the rapid
prototyping of reliable parallel programming systems. River is
implemented entirely in Python and is based on a few fundamental
concepts that enable the execution of code on multiple virtual machines
and provide a flexible mechanism for communication. These concepts are
supported by the River run-time system, which manages automatic
discovery, connection management, naming, process creation, and message
passing. The simplicity and elegance of the River core combined with
Python's dynamic typing and concise notation make it easy to rapidly
develop distributed applications and a variety of parallel run-time
systems for new programming models. 

%River introduces a novel dynamically
%typed message passing mechanism, called {\it super flexible messaging}
%that makes it easy to send arbitrary data as attribute-value pairs and
%can selectively receive based on matching attribute name and subsets of
%attribute values~\cite{Fedosov:2007:SFM}. This powerful mechanism
%eliminates the need to define a fixed packet structure or specify fixed
%remote interfaces.

% State management

To support different forms of fault tolerance based on state
redundancy~\cite{Gartner:1999:FT} we have developed an integrated state
management mechanism within River. The base mechanism supports
system-level and programmer directed checkpointing for arbitrary River
applications. Thus application-level modifications are not required, but
an explicit interface is provided if the application desires to control
the replication strategy. The base mechanism can be extended to support
various forms of state management, such as memory
exclusion~\cite{Plank:1999:CheckpointExclusion}, incremental
checkpointing~\cite{Plank:1995:Libckpt} or diskless
checkpointing~\cite{Plank:1998:CheckpointDiskless,Chen:2005:CheckpointCoding}.
The River state management facility can be used to support
checkpointing, load balancing, and whole or partial program migration in
clusters and grids. We have implemented our state management support on
top of Stackless
Python~\cite{Tismer:2000:StacklessPython,Stackless:2007}. We use
Stackless Python in an unconventional manner and designed a novel
system-call mechanism to ensure the separation of process state from the
rest of the River system.

% This separation is needed to fully detach
%River process state so that a Stackless tasklet can be pickled.

% GDB: Outline

This paper describes the design and implementation of the state management mechanism we have built in the River programming environment for distributed Python programs.
The rest of this paper is organized as follows.
Section~\ref{sec:Background} provides brief overviews of Python, Stackless Python, and River. 
Section~\ref{sec:Interface} describes the basic River interfaces for state management and some usage scenarios.
Section~\ref{sec:Implementation} details the implementation.
Section~\ref{sec:Experiments} presents the results of some basic experimentation of checkpointing Python applications.
Section~\ref{sec:Conclusions} makes some concluding remarks and gives
directions for future work.
